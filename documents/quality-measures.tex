\documentclass{article}
\usepackage{amsmath}

\DeclareMathOperator*{\argmax}{arg\,max}
\DeclareMathOperator*{\betadist}{Beta}

\usepackage{textcomp}\usepackage{xcolor}
\newcommand{\notaestesa}[2]{%
  \marginpar{\color{red!75!black}\textbf{\texttimes}}%
  {\color{red!75!black}%
    [\,\textbullet\,\textsf{\textbf{#1:}}%
    \textsf{\footnotesize#2}\,\textbullet\,]}%
}
\newcommand{\SC}[1]{\notaestesa{SC}{#1}}

\newcommand{\ie}{\textit{i.e.}}
\newcommand{\eg}{\textit{e.g.}}

\begin{document}

% genotype quality
%----------------------------------------------------------------------
\section{Genotype quality model}

We can model the genotype as the tossing of a coin.  In particular a
read could be either correct or incorrect, in the sense that a read
could either respect the genotype or it could be an error | where such
error would happen with probability $\epsilon$ | the technology error,
\eg, 15\% in the case of PacBio.  Hence, the model would represent a
coin toss where the coin is not fair, in particular we would have a
$(1-\epsilon)$ probability of obtaining a head (\ie, a correct read
according to the given genotype) and an $\epsilon$ chance of obtaining
a tail (\ie, an error in the read).
    
Now we consider a set of observations $D$ representing the reads
associated with a given genomic locus and denote as $\theta$ the
probability of obtaining a correct genotype (tossing an head in our
metaphor).  We want to obtain a $\theta$ that maximizes our
observations, thus we want the \emph{Maximum a posteriori} (MAP) value
for $\theta$:

\begin{equation}
  \begin{split}
    \theta_{MAP} &= \argmax_\theta P(\theta | D) \\
    &= \argmax_\theta P(D|\theta)P(\theta)
  \end{split}
\end{equation}

\noindent where the second equation is an expansion using Bayes's
theorem, and $P(D) = 1$, since we have the data.

Considering each term, we know that $P(D|\theta)$ is the likelihood
that the outcome is distributed according to a Bernoulli model:
$\theta^{\alpha_{C}} (1-\theta)^{\alpha_{E}}$, that is the probability
of obtaining $\alpha_{C}$ correct reads according to the genotype and
$\alpha_{E}$ errors in the reads.  Maximizing this value alone would
give us the \emph{Maximum likelihood estimation} (MLE), which is,
however, very dependent on the observations | indeed it would be
extremely unreliable in cases where the number of reads is low.  For
example, recalling our coin tossing scenario, if we saw $n$ heads and
no tail according to the MLE we would expect a tail in the $n+1$-st
toss with probability 0, even though this is clearly not the case.

Consider now the factor $P(\theta)$ which is the prior probability of
$\theta$, distributed according to a \textit{Beta} distribution, in
detail:
\begin{equation*}
  P(\theta) = \dfrac{\theta^{\beta_{C}-1} (1 - \theta)^{\beta_{E}-1}}
  {B(\beta_{C},\beta_{E})}
\end{equation*}
where $\beta_{C}$ and $\beta_{E}$ are respectively the number of
correct genotypes and errors according to a prior expectation and
$B(\beta_{C},\beta_{E})$ is the binomial distribution | therefore
$P(\theta)$ is distributed as a Beta distribution:
$P(\theta) \sim \betadist(\beta_{C},\beta_{E})$.

Since the conjugate prior of a binomial is a Beta distribution and
with some algebraic interpolation, whose details I will leave out
here, we have that
$P(\theta | D) \sim \betadist(\beta_{C} + \alpha_{C},\beta_{E} +
\alpha_{E})$ where $\alpha$ according to the number of observations
and the $\beta$ is according to the number expected given a prior.
In particular the mode of the MAP distribution is then:
\begin{equation}
  \theta_{MAP} = \dfrac{\beta_{C} + \alpha_{C} - 1}
  {\beta_{C} + \alpha_{C} + \beta_{E} + \alpha_{E} - 2}
  \label{eq:MAP}
\end{equation}

\subsection{Genotype quality measure}
Once we have acquired $\theta_{MAP}$ we have some option to compute
the quality of the genotype.  To obtain a quality of the genotype we
use a Z-test to measure the fitness of the sample according to our
null hypothesis.  The Z-test is used to evaluate a distribution that
can be approximated by a normal distribution.

We consider the null hypothesis $H_0$ as the hypothesis that the reads
come from a binomial distribution with probability $(1-\epsilon)$ of
being correct and $\epsilon$ of being errors.  We assume that this
probability is not fixed and can be modeled as a normal distribution
centered in $(1-\epsilon)$ with standard deviation $\sigma$, thus
$H_0 \sim \mathcal{N}(1-\epsilon, \sigma)$.  The meaning of this
approximation is that in a perfect world we should have exactly
$(1-\epsilon)$ probability of getting a correct read, but since the
world is not perfect this probability is not fixed and there is a
deviation of the error rate in different loci; while maintaining an
$\epsilon$ rate over all the genome.

We can then consider our sample at a given locus, with a probability
of having a correct read equal to $\theta_{MAP}$.  The z-score will be
calculated as follows, where $n$ is the coverage at the locus:
\begin{equation*}
  \mbox{z-score} = \dfrac{1-\epsilon - \theta_{MAP}}{\sigma / \sqrt{n}}
\end{equation*}
        
The quality measure would then be the corresponding p-value of the
z-score of the fitting of the observations to $H_0$.  In particular
this would mean that values that have a large amount of errors would
have a low quality, while loci with a smaller amount of errors than
expected would not be heavily penalized since this low expectation
would be mitigated by the prior probability.  Such p-value is denoted
as $G_{l_1}$ in the following.

\paragraph{Obtaining distribution values for the hypothesis}

To complete the definition we would need to obtain the values of the
standard deviation for $H_0$ and $H_1$.  Considering $H_0$ first we
can compute the standard deviation $\sigma_i$ of the Binomial
distribution for each locus $i$ by exploring the whole genome.  The
standard deviation would be calculated as
$\sigma_i = \sqrt{n_i (1-\epsilon) \epsilon}$ where $n_i$ is the
coverage at locus $i$; the standard deviation $\sigma$ of $H_0$ would
then be the mode or average of all the $\sigma_i$.  A similar
principle can be applied to obtain $\hat{\sigma}$ for each hypothesis
$H_1$ computed at each locus.

It is clear that $\sigma_i$ does not depend on the genotype but only
on the coverage, for this reason it is possible to compute it much
faster using a simulation, in fact we can generate $M$ samples, with
coverage $n_i \sim \mathcal{N}(\mu_c, \sigma_c)$, meaning that the
coverage of a simulation is drawn from a normal distribution with
standard deviation $\sigma_c$ and with mean equal to the average
coverage $\mu_c$ of the technology used.  In this way, it is not
necessary to scan the entire genome to obtain the coverage of every
locus, but they can be estimated using a prior assumption on the
technology used.

\paragraph{Obtaining prior values for the genotypes}

The most sensitive part would be to define the prior values for the
genotypes, \ie, values of $\beta_C$ and $\beta_E$ of
equation~\ref{eq:MAP}, since low values could be too heavily weighted
by the observations and, conversely, high values could overweigh the
observations.  One way to overcome this issue is to simulate a drawing
from $H_0$ of a total of $m$ samples, where $m$ is a user-defined
parameter.  According to this method $\beta_C$ and $\beta_E$ would be
the number of correct and errors, respectively, obtained in the
simulation over the total of $m$ samples from $H_0$.

% phasing quaility
%----------------------------------------------------------------------
\section{Phasing quality}

We consider two genomic loci $l_1$ and $l_2$ and their proposed
phasing $Q_{l_1,l_2}$; let $n$ be the number of reads that bridge the
two loci and $k$ the number of reads that support the phasing.  We
denote as $O$ the observations, \ie, the $n$ reads bridging $l_1$ and
$l_2$.  If we assume that $Q_{l_1,l_2}$ is correct that would mean
that the $k$ reads supporting the phasing are correct and the $(n-k)$
that oppose the phasing contain an error at $l_1$ or $l_2$ or both.

To obtain the probability that there is no error in the $k$ reads we
can start by calculating the probability that a read contains an error
in $l_1$ or $l_2$ or both as follows:

\begin{equation*}
  \begin{split}
    P(\mbox{error in read}) &= P(\mbox{error in }l_1  \vee \mbox{error in }l_2) \\
    &= P(\mbox{error in }l_1) + P(\mbox{error in }l_2) - P(\mbox{error in }l_1  \wedge \mbox{error in }l_2)\\
    &= \epsilon + \epsilon - \epsilon^2 = \epsilon(2-\epsilon)
  \end{split}
\end{equation*}

If we consider the whole set of reads that bridge both $l_1$ and
$l_2$, then the probability that there is no error in the $k$ reads
will be equal to $(1 - \epsilon(2-\epsilon))^k$.  To compute the
probability that there is an error in the other $n-k$ reads it will
simply be $(\epsilon(2-\epsilon))^{n-k}$, therefore:

\begin{equation}
  P(Q_{l_1,l_2}|O) = (1 - \epsilon(2-\epsilon))^k 
  (\epsilon(2-\epsilon))^{n-k}
  \label{eq:phasing-epsilon}
\end{equation}

It is possible to sophisticate the model by using the values computed
for the MAPs --- in fact we denote as $\theta_{MAP}(l_i)$ the MAP
computed for locus $i$ thus we can rewrite
equation~\ref{eq:phasing-epsilon}:

\begin{equation}
  P(Q_{l_1,l_2}|O) = (\theta_{MAP}(l_1)\theta_{MAP}(l_2))^k
  (1- \theta_{MAP}(l_1)\theta_{MAP}(l_2))^{n-k}
\end{equation}

% haplotyping quality
%----------------------------------------------------------------------
\section{Haplotyping quality}

To obtain the quality of the entire haplotyping we can combine the two
previous probabilities.  In particular we use the genotype
probabilities to weigh the phasing probabilities.  In detail it is the
probability of the genotype of $l_1$ and the probability of the
genotype of $l_2$ that function as weight for the probability of the
phasing, informally:

\begin{equation*}
  \begin{split}
    &\prod_{\forall l_1, l_2} P(\mbox{genotype } l_1 \wedge \mbox{ genotype } l_2) 
    P(\mbox{phasing of } l_1, l_2) = \\
    &\prod_{\forall l_1, l_2} P(\mbox{genotype } l_1) P(\mbox{ genotype } l_2) 
    P(\mbox{phasing of } l_1, l_2)
  \end{split}
\end{equation*}

Which will formally be equal to:

\begin{equation}
  \prod_{\forall l_1, l_2} G_{l_1} G_{l_2} P(Q_{l_1,l_2}|O)
\end{equation}

\end{document}
  
%  LocalWords:  Genotype genotype posteriori MLE genotypes MAPs
%  LocalWords:  overweigh
